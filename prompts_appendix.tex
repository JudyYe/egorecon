% Prompts Appendix Component
% This file can be included in an existing LaTeX document using: % Prompts Appendix Component
% This file can be included in an existing LaTeX document using: % Prompts Appendix Component
% This file can be included in an existing LaTeX document using: % Prompts Appendix Component
% This file can be included in an existing LaTeX document using: \input{prompts_appendix}
% 
% Required packages:
% - longtable
% - array
% - fancyvrb (or listings if preferred)

\section{System Prompt}

\begin{longtable}{|p{0.95\textwidth}|}
\hline
\textbf{System Instruction} \\
\hline
\begin{Verbatim}[fontsize=\small]
You are a precise visual classifier for hand-object contact detection in 
cluttered scenes.

CRITICAL CONSTRAINTS:
1. Each hand (left/right) can be in contact with AT MOST ONE object at a time.
2. "In contact" means direct physical touch: grasping, holding, pressing, or 
   any visible contact.
3. If a hand is not clearly touching any object, you must mark all objects as 0 
   for that hand.
\end{Verbatim} \\
\hline
\end{longtable}

\section{User Prompt}

\begin{longtable}{|p{0.95\textwidth}|}
\hline
\textbf{User Prompt Template} \\
\hline
\begin{Verbatim}[fontsize=\small]
Analyze this image for hand–object contact (actual touching, not just reaching).

VISUAL GUIDANCE:
The image has been annotated with colored masks:
- GREEN mask = Left hand
- RED mask = Right hand
- Other COLORED masks = Candidate objects (each object has a unique color)

CANDIDATE OBJECTS (in order):
1. obj1
2. obj2
3. obj3
...

STRICT DEFINITION OF CONTACT:
For this task, contact means clear physical touching in this frame only.

Contact (label = 1) requires BOTH:
1. Mask intersection:
   - The hand mask and the object mask share some pixels or directly overlap 
     at the boundary (no visible gap).
2. Touching region:
   - The overlap is at a plausible touching area (finger tips, fingers, palm, 
     side of hand) on the visible surface of the object.

NO Contact (label = 0) in all of these cases:
- The hand is reaching toward, hovering above, or very close to an object with 
  a visible gap between masks.
- The hand is aligned in depth (e.g., above or behind the object) but the masks 
  do not intersect.
- The hand is in a pose that suggests future contact, but there is no current 
  touching in this single frame.
- There is only a tiny, ambiguous intersection (1–2 pixels) that could be noise 
  or occlusion. In such uncertain cases, choose 0 (no contact).

IMPORTANT:
- **Reaching or hovering is NOT contact.**
- **If you are unsure whether contact is happening, choose 0 (no contact).**

CONSTRAINTS (VALIDATION CHECK):
- Each hand can touch AT MOST ONE object.
  - Sum of left across all objects must be ≤ 1.
  - Sum of right across all objects must be ≤ 1.
- If a hand is not clearly touching any object, it should have 0 for all objects.

OUTPUT FORMAT:
Return only a JSON object in this exact format (no extra text):

{
  "obj1": {"left": 0, "right": 1},
  "obj2": {"left": 0, "right": 0},
  "obj3": {"left": 1, "right": 0}
}

Where:
- 1 = the specified hand is clearly touching that object in this frame.
- 0 = the specified hand is not touching that object in this frame.
\end{Verbatim} \\
\hline
\end{longtable}

\section{In-Context Learning (ICL) Prompt}

\begin{longtable}{|p{0.95\textwidth}|}
\hline
\textbf{ICL Example Format (User Message)} \\
\hline
\begin{Verbatim}[fontsize=\small]
Example 1 (Ground Truth)
Sequence 002043, frame 0085
Candidate objects:
1. 000016
2. 000008
Respond only with the JSON contact labels for this frame.
[Overlay Image]
\end{Verbatim} \\
\hline
\textbf{ICL Example Format (Assistant Response)} \\
\hline
\begin{Verbatim}[fontsize=\small]
{
  "000016": {
    "left": 0,
    "right": 1
  },
  "000008": {
    "left": 0,
    "right": 0
  }
}
\end{Verbatim} \\
\hline
\textbf{ICL Prompt Structure} \\
\hline
Each ICL example consists of:
\begin{enumerate}
    \item \textbf{User message}: Contains example header, sequence information, 
          object list, instruction to respond with JSON, and the overlay image 
          showing colored hand/object masks.
    \item \textbf{Assistant response}: Contains the ground truth JSON contact 
          labels for that frame.
\end{enumerate}

Multiple examples can be chained before the actual query. The final query uses 
the same user prompt template as described in Section 2, followed by the target 
overlay image. \\
\hline
\end{longtable}

% 
% Required packages:
% - longtable
% - array
% - fancyvrb (or listings if preferred)

\section{System Prompt}

\begin{longtable}{|p{0.95\textwidth}|}
\hline
\textbf{System Instruction} \\
\hline
\begin{Verbatim}[fontsize=\small]
You are a precise visual classifier for hand-object contact detection in 
cluttered scenes.

CRITICAL CONSTRAINTS:
1. Each hand (left/right) can be in contact with AT MOST ONE object at a time.
2. "In contact" means direct physical touch: grasping, holding, pressing, or 
   any visible contact.
3. If a hand is not clearly touching any object, you must mark all objects as 0 
   for that hand.
\end{Verbatim} \\
\hline
\end{longtable}

\section{User Prompt}

\begin{longtable}{|p{0.95\textwidth}|}
\hline
\textbf{User Prompt Template} \\
\hline
\begin{Verbatim}[fontsize=\small]
Analyze this image for hand–object contact (actual touching, not just reaching).

VISUAL GUIDANCE:
The image has been annotated with colored masks:
- GREEN mask = Left hand
- RED mask = Right hand
- Other COLORED masks = Candidate objects (each object has a unique color)

CANDIDATE OBJECTS (in order):
1. obj1
2. obj2
3. obj3
...

STRICT DEFINITION OF CONTACT:
For this task, contact means clear physical touching in this frame only.

Contact (label = 1) requires BOTH:
1. Mask intersection:
   - The hand mask and the object mask share some pixels or directly overlap 
     at the boundary (no visible gap).
2. Touching region:
   - The overlap is at a plausible touching area (finger tips, fingers, palm, 
     side of hand) on the visible surface of the object.

NO Contact (label = 0) in all of these cases:
- The hand is reaching toward, hovering above, or very close to an object with 
  a visible gap between masks.
- The hand is aligned in depth (e.g., above or behind the object) but the masks 
  do not intersect.
- The hand is in a pose that suggests future contact, but there is no current 
  touching in this single frame.
- There is only a tiny, ambiguous intersection (1–2 pixels) that could be noise 
  or occlusion. In such uncertain cases, choose 0 (no contact).

IMPORTANT:
- **Reaching or hovering is NOT contact.**
- **If you are unsure whether contact is happening, choose 0 (no contact).**

CONSTRAINTS (VALIDATION CHECK):
- Each hand can touch AT MOST ONE object.
  - Sum of left across all objects must be ≤ 1.
  - Sum of right across all objects must be ≤ 1.
- If a hand is not clearly touching any object, it should have 0 for all objects.

OUTPUT FORMAT:
Return only a JSON object in this exact format (no extra text):

{
  "obj1": {"left": 0, "right": 1},
  "obj2": {"left": 0, "right": 0},
  "obj3": {"left": 1, "right": 0}
}

Where:
- 1 = the specified hand is clearly touching that object in this frame.
- 0 = the specified hand is not touching that object in this frame.
\end{Verbatim} \\
\hline
\end{longtable}

\section{In-Context Learning (ICL) Prompt}

\begin{longtable}{|p{0.95\textwidth}|}
\hline
\textbf{ICL Example Format (User Message)} \\
\hline
\begin{Verbatim}[fontsize=\small]
Example 1 (Ground Truth)
Sequence 002043, frame 0085
Candidate objects:
1. 000016
2. 000008
Respond only with the JSON contact labels for this frame.
[Overlay Image]
\end{Verbatim} \\
\hline
\textbf{ICL Example Format (Assistant Response)} \\
\hline
\begin{Verbatim}[fontsize=\small]
{
  "000016": {
    "left": 0,
    "right": 1
  },
  "000008": {
    "left": 0,
    "right": 0
  }
}
\end{Verbatim} \\
\hline
\textbf{ICL Prompt Structure} \\
\hline
Each ICL example consists of:
\begin{enumerate}
    \item \textbf{User message}: Contains example header, sequence information, 
          object list, instruction to respond with JSON, and the overlay image 
          showing colored hand/object masks.
    \item \textbf{Assistant response}: Contains the ground truth JSON contact 
          labels for that frame.
\end{enumerate}

Multiple examples can be chained before the actual query. The final query uses 
the same user prompt template as described in Section 2, followed by the target 
overlay image. \\
\hline
\end{longtable}

% 
% Required packages:
% - longtable
% - array
% - fancyvrb (or listings if preferred)

\section{System Prompt}

\begin{longtable}{|p{0.95\textwidth}|}
\hline
\textbf{System Instruction} \\
\hline
\begin{Verbatim}[fontsize=\small]
You are a precise visual classifier for hand-object contact detection in 
cluttered scenes.

CRITICAL CONSTRAINTS:
1. Each hand (left/right) can be in contact with AT MOST ONE object at a time.
2. "In contact" means direct physical touch: grasping, holding, pressing, or 
   any visible contact.
3. If a hand is not clearly touching any object, you must mark all objects as 0 
   for that hand.
\end{Verbatim} \\
\hline
\end{longtable}

\section{User Prompt}

\begin{longtable}{|p{0.95\textwidth}|}
\hline
\textbf{User Prompt Template} \\
\hline
\begin{Verbatim}[fontsize=\small]
Analyze this image for hand–object contact (actual touching, not just reaching).

VISUAL GUIDANCE:
The image has been annotated with colored masks:
- GREEN mask = Left hand
- RED mask = Right hand
- Other COLORED masks = Candidate objects (each object has a unique color)

CANDIDATE OBJECTS (in order):
1. obj1
2. obj2
3. obj3
...

STRICT DEFINITION OF CONTACT:
For this task, contact means clear physical touching in this frame only.

Contact (label = 1) requires BOTH:
1. Mask intersection:
   - The hand mask and the object mask share some pixels or directly overlap 
     at the boundary (no visible gap).
2. Touching region:
   - The overlap is at a plausible touching area (finger tips, fingers, palm, 
     side of hand) on the visible surface of the object.

NO Contact (label = 0) in all of these cases:
- The hand is reaching toward, hovering above, or very close to an object with 
  a visible gap between masks.
- The hand is aligned in depth (e.g., above or behind the object) but the masks 
  do not intersect.
- The hand is in a pose that suggests future contact, but there is no current 
  touching in this single frame.
- There is only a tiny, ambiguous intersection (1–2 pixels) that could be noise 
  or occlusion. In such uncertain cases, choose 0 (no contact).

IMPORTANT:
- **Reaching or hovering is NOT contact.**
- **If you are unsure whether contact is happening, choose 0 (no contact).**

CONSTRAINTS (VALIDATION CHECK):
- Each hand can touch AT MOST ONE object.
  - Sum of left across all objects must be ≤ 1.
  - Sum of right across all objects must be ≤ 1.
- If a hand is not clearly touching any object, it should have 0 for all objects.

OUTPUT FORMAT:
Return only a JSON object in this exact format (no extra text):

{
  "obj1": {"left": 0, "right": 1},
  "obj2": {"left": 0, "right": 0},
  "obj3": {"left": 1, "right": 0}
}

Where:
- 1 = the specified hand is clearly touching that object in this frame.
- 0 = the specified hand is not touching that object in this frame.
\end{Verbatim} \\
\hline
\end{longtable}

\section{In-Context Learning (ICL) Prompt}

\begin{longtable}{|p{0.95\textwidth}|}
\hline
\textbf{ICL Example Format (User Message)} \\
\hline
\begin{Verbatim}[fontsize=\small]
Example 1 (Ground Truth)
Sequence 002043, frame 0085
Candidate objects:
1. 000016
2. 000008
Respond only with the JSON contact labels for this frame.
[Overlay Image]
\end{Verbatim} \\
\hline
\textbf{ICL Example Format (Assistant Response)} \\
\hline
\begin{Verbatim}[fontsize=\small]
{
  "000016": {
    "left": 0,
    "right": 1
  },
  "000008": {
    "left": 0,
    "right": 0
  }
}
\end{Verbatim} \\
\hline
\textbf{ICL Prompt Structure} \\
\hline
Each ICL example consists of:
\begin{enumerate}
    \item \textbf{User message}: Contains example header, sequence information, 
          object list, instruction to respond with JSON, and the overlay image 
          showing colored hand/object masks.
    \item \textbf{Assistant response}: Contains the ground truth JSON contact 
          labels for that frame.
\end{enumerate}

Multiple examples can be chained before the actual query. The final query uses 
the same user prompt template as described in Section 2, followed by the target 
overlay image. \\
\hline
\end{longtable}

% 
% Required packages:
% - longtable
% - array
% - fancyvrb (or listings if preferred)

\section{System Prompt}

\begin{longtable}{|p{0.95\textwidth}|}
\hline
\textbf{System Instruction} \\
\hline
\begin{Verbatim}[fontsize=\small]
You are a precise visual classifier for hand-object contact detection in 
cluttered scenes.

CRITICAL CONSTRAINTS:
1. Each hand (left/right) can be in contact with AT MOST ONE object at a time.
2. "In contact" means direct physical touch: grasping, holding, pressing, or 
   any visible contact.
3. If a hand is not clearly touching any object, you must mark all objects as 0 
   for that hand.
\end{Verbatim} \\
\hline
\end{longtable}

\section{User Prompt}

\begin{longtable}{|p{0.95\textwidth}|}
\hline
\textbf{User Prompt Template} \\
\hline
\begin{Verbatim}[fontsize=\small]
Analyze this image for hand–object contact (actual touching, not just reaching).

VISUAL GUIDANCE:
The image has been annotated with colored masks:
- GREEN mask = Left hand
- RED mask = Right hand
- Other COLORED masks = Candidate objects (each object has a unique color)

CANDIDATE OBJECTS (in order):
1. obj1
2. obj2
3. obj3
...

STRICT DEFINITION OF CONTACT:
For this task, contact means clear physical touching in this frame only.

Contact (label = 1) requires BOTH:
1. Mask intersection:
   - The hand mask and the object mask share some pixels or directly overlap 
     at the boundary (no visible gap).
2. Touching region:
   - The overlap is at a plausible touching area (finger tips, fingers, palm, 
     side of hand) on the visible surface of the object.

NO Contact (label = 0) in all of these cases:
- The hand is reaching toward, hovering above, or very close to an object with 
  a visible gap between masks.
- The hand is aligned in depth (e.g., above or behind the object) but the masks 
  do not intersect.
- The hand is in a pose that suggests future contact, but there is no current 
  touching in this single frame.
- There is only a tiny, ambiguous intersection (1–2 pixels) that could be noise 
  or occlusion. In such uncertain cases, choose 0 (no contact).

IMPORTANT:
- **Reaching or hovering is NOT contact.**
- **If you are unsure whether contact is happening, choose 0 (no contact).**

CONSTRAINTS (VALIDATION CHECK):
- Each hand can touch AT MOST ONE object.
  - Sum of left across all objects must be ≤ 1.
  - Sum of right across all objects must be ≤ 1.
- If a hand is not clearly touching any object, it should have 0 for all objects.

OUTPUT FORMAT:
Return only a JSON object in this exact format (no extra text):

{
  "obj1": {"left": 0, "right": 1},
  "obj2": {"left": 0, "right": 0},
  "obj3": {"left": 1, "right": 0}
}

Where:
- 1 = the specified hand is clearly touching that object in this frame.
- 0 = the specified hand is not touching that object in this frame.
\end{Verbatim} \\
\hline
\end{longtable}

\section{In-Context Learning (ICL) Prompt}

\begin{longtable}{|p{0.95\textwidth}|}
\hline
\textbf{ICL Example Format (User Message)} \\
\hline
\begin{Verbatim}[fontsize=\small]
Example 1 (Ground Truth)
Sequence 002043, frame 0085
Candidate objects:
1. 000016
2. 000008
Respond only with the JSON contact labels for this frame.
[Overlay Image]
\end{Verbatim} \\
\hline
\textbf{ICL Example Format (Assistant Response)} \\
\hline
\begin{Verbatim}[fontsize=\small]
{
  "000016": {
    "left": 0,
    "right": 1
  },
  "000008": {
    "left": 0,
    "right": 0
  }
}
\end{Verbatim} \\
\hline
\textbf{ICL Prompt Structure} \\
\hline
Each ICL example consists of:
\begin{enumerate}
    \item \textbf{User message}: Contains example header, sequence information, 
          object list, instruction to respond with JSON, and the overlay image 
          showing colored hand/object masks.
    \item \textbf{Assistant response}: Contains the ground truth JSON contact 
          labels for that frame.
\end{enumerate}

Multiple examples can be chained before the actual query. The final query uses 
the same user prompt template as described in Section 2, followed by the target 
overlay image. \\
\hline
\end{longtable}
